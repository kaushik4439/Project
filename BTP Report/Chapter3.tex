\section{Introduction}
In order to secure the cloud against the varied security threats and attacks like: SQL injection, Cross Site Scripting (XSS) attacks, DoS and DDoS attacks, Google Hacking and compelled Hacking, different cloud service suppliers adopt completely different techniques. a number of normal techniques so as to observe the above mentioned attacks area unit as: Avoiding the usage of dynamically generated
SQL within the code, finding the meta-structures employed in the code, confirmative all user entered parameters, disallowing and removal of unwanted knowledge and characters, etc. A generic security framework must be found out for AN optimized value performance magnitude relation. the most criterion to be stuffed up by the generic security framework are to interface with any style of cloud atmosphere, and to be ready to handle and observe predefined as well as bespoke security policies. an identical approach is getting used by Symantec Message LabsWeb Security cloud that blocks the protection threats originating from web and filters the info
before they reach the network. net security clouds security design rests on 2 components:
\textbf{a. Multi layer security:} so as to confirm that knowledge security and block potential malwares, it consists of multi-layer security and therefore a robust security platform.
\textbf{b. URL address filtering:} it's being determined that the attacks area unit launched through numerous web content and websites and therefore filtering of the web-pages, ensures that no such harmful or threat carrying web page gets accessible. Also, content from undesirable sites will be blocked. With its
elastic technology, it provides security even in extremely conflicting environments and ensures protection against new and convergency malware threats.
\section{Man In Middle attacks :}
If secure socket layer (SSL) is not properly configured, then any attacker is able to access the data exchange between two parties. In Cloud, an attacker is able to access the data communication among data centers.

\textbf{Mitigation:} Legitimate SSL setup and information correspondence tests between approved gatherings can be valuable to lessen the danger of Man-in- the-Middle assault.
\section{DoS and DDoS attacks : }
The symptoms to a DoS or DDoS attack are: system speed gets reduced and programs run very slowly, large number of connection requests from a large number of users, less number of available resources. Although when launched in full strength DDoS attacks are very harmful as they exhaust all the network resources. 

\textbf{Mitigation:} A careful monitoring of the network can help in keeping these attacks in control. 
\section{IP spoofing attacks: }
In case of IP spoofing an attacker tries to spoof the users that the packets are coming from reliable sources. Thus the attacker takes control over the client’s data or system showing himself as the trusted party.

\textbf{Mitigation:} Spoofing attacks can be checked by using encryption techniques and performing user authentication based on Key exchange. Techniques like IPSec do help in mitigating the risks of spoofing. By enabling encryption sessions and performing filtering at the incoming and outgoing entrances spoofing attacks can be reduced.
\section{Hypervisor attacks: }
VM-based rootkits initiate a hypervisor compromising the existing host OS to a VM.
The new guest OS assumes that it is running as the host OS with the corresponding control over the resources,however, in reality this host does not exist. Hypervisor
also creates a covert channel to execute unauthorized code into the system. This allows an attacker to control over any VM running on the host machine and to manipulate
the activities on the system.

\textbf{Mitigation:} The threat arising due to VM-Level
vulnerabilities can be mitigated by monitoring through
IDS (Instruction Detection System)/IPS (Intrusion
Prevention System) and by implementing firewall.