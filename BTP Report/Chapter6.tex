
Deep learning in security learns real face (attack or legitimate) of cyber information on even little
variations or changes, indicating the resiliency of deep learning to little changes in network information by
creating high level invariant representations of the training information. Machine learning refers to systems
that  measure ready to mechanically improve with expertise. historically, in spite of what number times you
use system to perform an equivalent actual task, the system won’t get any smarter. Whenever we launch our
browser and visit an equivalent actual website? a conventional browser won’t ”learn” that it ought to most likely
just bring you there by itself once 1ist launched. With ML, package will gain the power to be told from
previous observations to form inferences regarding  future behavior, moreover as guess what you would like to
do in new situations.  However to understand how ML works we tend to 1ist have to be compelled to understand the fuel that creates ML (data)
possible: information(data). think about associate email spam detection algorithmic rule. Original spam filters would merely blacklist
certain addresses and permit alternative mail through. ML increased this significantly by scrutiny verified
spam emails with verified legitimate email and seeing that ”features” were gift additional oft in
one or the opposite. as an example, designedly misspelled words,the presence of hyperlinks
to best-known malicious websites, and virus-link attachments  measure doubtless options indicative of spam rather
than legitimate email. (More discussion on ”features” below.) This method of mechanically inferring a
label (i.e., ”spam” vs ”legitimate”) is named classification, and is one amongst the foremost applications of ML
techniques. it's price mentioning that one alternative quite common technique is prognostication, the use of
historical information to predict future behavior.

\section{security using Deep learning}
DL has many variants whose quality depends on the character of the appliance . Autoencoder
and its families are the foremost applied  models, and have resulted in promising leads to unattended
learning. Autoencoder map input options a similar range of output options, minimizing
reconstruction errors.In different word, given a collection of untagged training knowledge x(1), x(2), x(3), , stacked
autoencoder maps to output options to be up to the input options (i.e., y(i) = x(i)). Technically, autoencoding
is a compress-decompress technique of pattern extraction from knowledge. The drawbacks of classical
machine learning in attack detection and lack of automatic feature engineering, low detection rate, and incapability of detection tiny mutants of existing attacks and zero-day attacks. These limitations will
be improved by adopting DL. By making high-level representations of options, DL discovers advanced
functions that map input to output without manual intervention by consultants. DL additionally exploits the advantages
of automatic hierarchic feature learning from data. Attack detection may gain advantage from a
pre-training theme of stacked autoencoders for automatic feature learning. a way to use stacked
autoencoder is to train a model with a mixture of normal/attack sample of the untagged network so the
model identifies patterns of attack and traditional knowledge by a self-learning theme. The detected patterns 
mapped to labelled check knowledge as attack and traditional.

and normal. 