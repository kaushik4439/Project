Cloud computing is associate degree rising paradigm for big scale infrastructures.It
has the advantage of decreasing expense by sharing process and capability assets, joined with associate degree on-request provisioning component reckoning on a compensation for each utilization arrange of action.But the chief concern in cloud environments is to supply security around multi-tenancy and isolation.Security at totally different levels such as Network level, Host level and Application level is critical to stay the overcloud and running continuously. 

The security downside becomes additional difficult beneath the cloud models as new
dimensions have entered into the matter scope associated with the model design,multi-tenancy, snap and layer dependency stack. we tend to introduce a close analysis of the cloud security downside. we tend to investigated the problem from the cloud design perspective.Also provided potential measures to urge prevented from
these attacks exploitation numerous security technique and within the last half we
tend to use NESSI2 package and showed the simulation for DDOS attack and generated its report, graphs and wrote conclusion.

Fog computing is an architecture which end-user clients or nearby edge devices to carry out a considerable amount of storage , communication
, control , measurement and management rather than controlled primarily by network gateways.In fog computing, for any instance of time same
edge device can be used by multiple smart applications with different set of users which raises the issue of security of edge device.Attacks in fog computing
can be classified into two main categories (a) unauthenticated attacks and (b) unauthorized attacks . Attack detection in fog computing is done by using 
machine learning algorithms , we use machine learning because it automatically improves with experince and enchance its capabilites .Attack detection could benefit from a
pre-training scheme of stacked autoencoders for automatic feature learning. One way to apply stacked
autoencoder is to train a model with a mix of normal/attack sample of the unlabeled network so that the
model identifies patterns of attack and normal data by a self-learning scheme. The detected patterns are
mapped to labeled test data as attack and normal.

Our work is to design and implement deep learning based mechanism this work has used self taught deep learning scheme in which unsupervised feature learning has
been employed on training data.The learnt features were applied to the labeled test dataset for classification into attack and normal .
We used NSL-KDD intrusion dataset which is available in csv format for model validation and
evaluations. The original dataset consists of 125,973 records of train and 22,544 records of test, each
with 41 features such as duration,protocol, service, flag, source bytes, destination bytes, etc.
We have used k-fold cross validation to check the accuracy of our model and generated its report and wrote conclusions. 

